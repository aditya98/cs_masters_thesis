\chapter{Simulation Design}
\label{cap:simulationdesign}

% To study the impact of \ac{MV} used with \apufs on the stability of \apuf we simulate the \apuf, the \mpuf, and the \mxpuf.
To study the impact of \ac{MV} on the stability of delay based \pufs we simulate different versions of \apufs combined with \ac{MV}.
Based on them the influence of \ac{MV} on the only relevant attack on large \xpufs, as pointed out in Sec. \ref{sec:essentialattacks}, is studied.

\todo{tech spec}

For a more detailed explanation of the simulations and the attacks this section first gives a description of the simulation of a \apuf that is basis for all other \puf simulations in this thesis.
Afterwards, the principles used by the different attacks on \mpufs and \mxpufs are described.

% all training sets and test sets are random generated by pseudo-random numbers generator libraries.
% python numpy
% hpc up to 700 cores simultaniously 
% mathematica plots
%========================================
% -why the cma-es attack, ref to attack sec -> only attack which scales linear with increasing k 


\section{\ac{PUF} Design}
\label{sec:pufsimulation}

The \apuf simulations are based on the \apuf representation explained in Sec. \ref{sec:theoretical}.

% old:
% They use randomly chosen normally distributed values with zero mean and given variance for delay difference values $\delta_i$ that represent the \apuf.
% % https://en.wikipedia.org/wiki/Normal_distribution
% This is a common approach for natural randomly occurring deviations whose distribution is unknown, since the sum of independent random variable, regardless of their distribution, tend to toward a normal distribution known as the central limit theorem \cite{2017CentralTheorem}. %+
The sum of independent random variable, regardless of their distribution, tend to toward a normal distribution, known as the central limit theorem \cite{2017CentralTheorem}.
Hence, it is a common approach for natural randomly occurring deviations, whose distribution is unknown, to be chosen of a normal distribution.
For that reason, the delay difference values $\delta_i$, which represent the variation of the \apuf due to the manufacturing process, are randomly chosen normally distributed values with zero mean and given variance. %+

The different noise values added in Eq. (\ref{equ:pufresponse}) are also normally distributed random values with zero mean and given variance.
The variance of the stage noise distribution, explained in Sec. \ref{sec:theoretical}, depends on the size $\gls{n}$ of the \apuf, as required by Eq. (\ref{equ:stagenoisedistribution}).

%old: The distribution of the delay difference values and all noise values depend for a physical \apuf implementation on the used electronic components, as shown in Sec. \ref{sec:physical}.%+
For a physical \apuf implementation, the distribution of the delay difference values and all noise values depend on the used electronic components, as shown in Sec. \ref{sec:physical}.
The ratio between $\sigmaSNoise$ and $\sigmaModel$ and the value of $\sigmaANoise$ can affect the stability of the challenges of the \apuf by promoting wrong \puf evaluations if the noise is relatively high.
As this work studies the impact of adding \ac{MV} to \apufs in a comparison, the real value of the ratio does not matter to this study.
The impact of changing the ratio is not part of this work and could be the subject of another thesis. 
However, it crucial to not change the ratio during the simulation to obtain results that are comparable.

The response of the \apuf model is simulated by Eqs. (\ref{equ:pufmodelc}) or (\ref{equ:pufmodelw}) before the noise values are added.
All other implementations of \puf concepts used in this thesis, e.g. the \mpuf, are based on these \apuf equations.
For the attack simulation of the \ac{CMA-ES} attack Eq. (\ref{equ:pufmodelw}) is used since the delay vector $\gls{w}$ is approximated, as explained in Sec. \ref{sec:machinelearningdesign}.

%========================================
% -normal dist to choose delay difference values
% -Noise added by normal distributed values as in equ.
% -stage Noise got scaled by sqrt of n
% -addition arbiter noise added
% -Sigma Model to sigma noise relation + Arbiter noise 
% --> this relates on the electrical components used for the implementation
% --> as this work is a comparison it is crucial not to change the relation to gain results that are comparable!
% (+ set Ref in Chap 7 beginning)
% -response calculated by equ bla und bla ref to math section
% -For the attack equ bla is used as the weight vector w is approximated described in sec \ref{sec:machinelearningdesign}

\section{Machine Learning Design}
\label{sec:machinelearningdesign}

% To attack \mpufs and \mxpufs, as described in Chap. \ref{cap:majorityarbiter}, the reliability based \ac{CMA-ES} attack introduced by Becker in his paper with the title "The Gap Between Promise and Reality: On the Insecurity of XOR Arbiter PUFs" and further sources are used \cite{Becker2015ThePUFs,2017CMA-ES,Hansen2011TheTutorial,Hansen2006TheReview}. %+
To attack \mpufs and \mxpufs, as described in Chap. \ref{cap:majorityarbiter}, the reliability based \ac{CMA-ES} attack introduced by Becker is interesting to study \cite{Becker2015ThePUFs}.
Further sources have been used \cite{Becker2015ThePUFs,2017CMA-ES, Hansen2011TheTutorial, Hansen2006TheReview}. %+
It is the only relevant attack that can successfully create well performing models of large \xpufs, as described in Sec. \ref{sec:essentialattacks}.

As large \xpufs face some stability challenges, the concept of \ac{MV} is used to overcome them, as described in Sec. \ref{sec:majorityxorarbiter}.
Hence, the attack is applied to \apufs and \xpufs with and without the concept of \ac{MV} to study its impact on the attack. %+

% The attack implementations follow the description of the paper of Becker et al. to be able to reproduce their results of well performing models of large \xpufs \cite{Becker2015ThePUFs}.
The attack implementation is based on the original source \cite{Becker2015ThePUFs}.
A model performs well if it correctly predicts a high proportion of the challenges from a test set.

%old: The reliability based \ac{CMA-ES} algorithm relies on the \ac{CMA-ES} algorithm, which is described in Sec. \ref{sec:cma-es}.
The reliability based \ac{CMA-ES} algorithm relies on the principles of the classic \ac{CMA-ES} algorithm, which is described in Sec. \ref{sec:cma-es}.
To provide a better understanding of how the attack works, this section outlines the applied principles and the attack implementations. %+
First, the reliability approach, used in a fitness function, to rate the models created by the algorithm is explained.
After that, it is shown how the \ac{CMA-ES} algorithm uses this to approximates well performing models of the attacked \apuf.
Finally, we describe how to successfully apply the algorithm to attack \xpufs.

%========================================

\subsection{Reliability}
\label{sec:reliability}
\todo{maybe rename to "Fitness function"}

The \ac{ES} algorithm is inspired by the natural evolution process of survival of the fittest (see Sec. \ref{sec:evolutionstrategies}) and thus needs a method to score the fitness of a trained model.
This function is called fitness function.
In Sec. \ref{sec:cma-esdesign}, we will outline how the fitness function is used in the \ac{CMA-ES} algorithm.

% Becker's fitness function computes the Pearson correlation coefficient between the reliability values $\gls{h}_i$ and the hypothetical reliability values $g_i$ of all challenges of the training set.
% unused below:
% Becker's fitness function computes the Pearson correlation coefficient between the reliability values $\gls{h}_i$ and the hypothetical reliability values $g_i$ of all challenges of the training set.
% This is done to approximate the attacked \puf by adapting its reliability behavior of the responses of every challenge of the training set. %+
% idea: Von der Stabilität kann auf den konkreten Wert des Signalunterschieds (delay difference) geschlossen werden

Becker's fitness function uses reliability values of challenges that are evaluated by the attacked \puf.
To measure the reliability value $h_i$, which is defined in the next paragraph, of a challenge, the challenge has to be evaluated multiple times by the attacked \puf.
Hence, the attack requires the possibility to freely evaluate challenges similar to a chosen-plaintext attack \cite{2016Chosen-plaintextAttack}.
This process of measuring multiple challenge evaluations has to be done only once, since the same values are used during the complete attack. %+ % the reliability values of the challenges of the attacked \puf
The set of multiple evaluated challenges is in the case of the reliability based \ac{CMA-ES} attack the training set with $v$ number of challenges.
The test set is created by single evaluations of different challenges, as specified in Cap. \ref{cap:mla}.


% Against the definition of stability \ref{equ:stability}, in terms of the reliability based \ac{CMA-ES} attack, the reliability of a challenge and the hypothetical reliability of a challenge have both different definitions than the stability, as described in the next paragraphs. 
% First an explanation for the reliability and the hypothetical reliability is given before it is described how they are used to rate the models.

% To measure the reliability value $h_i$ of a challenge it has to be evaluated multiple times by the attacked \puf.
% This process of measuring the reliability values of the challenges of the attacked \puf has to be done only once as the same values are used during the complete attack. %+

% old: Against the definition of stability in Eq. (\ref{equ:stability}), in terms of the reliability based \ac{CMA-ES} attack, the reliability of a challenge has a different definition than the stability.
How the definition of reliability differs from the definition of stability that is given in Eq. (\ref{equ:stability}), in terms of the reliability based \ac{CMA-ES} attack, is highlighted now.
% old: We define the number of evaluation $\gls{j}$ to be the number of times a challenge is evaluated by the same \puf and its responses $\gls{r}$ captured to calculate it reliability $\gls{h}_i$.
Let $\gls{j}$ be the number of times a challenge is evaluated by the same \puf and let $\gls{r}_i$, $1 \le i \le j$ be the responses captured to calculate the reliability $\gls{h}_i$.
The reliability $\gls{h}_i$ of a challenge $\gls{c}$ is computed of its responses $r_1, r_2, ..., r_j$ as follows \cite{Becker2015ThePUFs}:

\begin{align}
h_i &= |\frac{j}{2} - \sum_{l = 1}^{j}r_l| \label{equ:reliability}
\end{align}
% Nils: In CS language: this i shadows the i that is already declared in the outside scope. In math, shadowing is considered an error, not just an inconvinience.


%old: In Eq. (\ref{equ:reliability}) it can be seen that the reliability value of a challenge $c$ has its maximum of $\frac{j}{2}$ when the \puf evaluates with the same response $r$, either $0$ or $1$, for all $j$ times of evaluations. %+
In Eq. (\ref{equ:reliability}) it can be seen that the reliability value of a challenge $c$ has its maximum of $\frac{j}{2}$ when the \puf evaluates with all $\gls{r}_i$, $i \le j$, are simultaneously $0$ or $1$, for all $j$ times of evaluations. %+
A challenges that evaluates one or more times of $j$ evaluations to a different response as the other responses has a lowered reliability and is called unreliable challenge.
Hence $h = (h_1, ..., h_v)$ is the reliability vector of $v$ challenges from the training set evaluated by the attacked \puf.

Becker's fitness function uses also the hypothetical reliability values of challenges evaluated by the modeled \pufs.
The hypothetical reliability of a challenge has also a different definition than the reliability and the stability, in terms of the reliability based \ac{CMA-ES} attack.
% After that, the hypothetical reliability values $g_i$ of the models created by the algorithm have to be calculated.
To rate all models, hypothetical reliability values have to be calculated for every model created by the algorithm.
The models are defined by their delay vectors $\gls{w}$ of real-valued numbers and of size $\gls{n} + 1$, as used in the \apuf model representation Eq. (\ref{equ:pufmodelw}), and are known by the algorithm. %+
Additionally, the feature vector $\gls{x}_{n+1}$ of real-valued numbers, which is defined in Sec. \ref{sec:theoretical}, is computed from every challenge of the training set by Eq. (\ref{equ:featurevector}).
Hence the vectors $w$, the feature vectors $\gls{x}$, and the \apuf model Eq. (\ref{equ:pufmodelw}) are used to calculate the hypothetical reliability $g_i$ for a challenge as follows:

% Nils: when introducing new variables, like x here, or w in the sentence before, always give the type. For instance, x is a vector of real-valued numbers of size n+1 (?). You can do that either in math: x\in\mathbb{R}^{n+1} or in English (see above). Here you are using the dot product of two vectors that you did not properly declare. This is why it is important for the reader to know the vector types and sizes, otherwise the dot product cannot be understood by the reader.
\begin{equation}
\begin{aligned}
g_i &=
\begin{cases}
1,\ \ \text{if}\ |\langle w, x \rangle| > \epsilon\\
0,\ \ \text{if}\ |\langle w, x \rangle| < \epsilon \label{equ:hypotheticalreliability}
\end{cases}
\end{aligned}
\end{equation}

The hypothetical reliability categorizes every challenge for the model defined by $w$ into hypothetical reliable and hypothetical unreliable. %+
The challenge is defined to be hypothetical reliable when its delay difference computed by $\langle w, \gls{x} \rangle$ is greater than the error boundary $\epsilon$ and results in $g_i = 1$.
The challenge is defined to be hypothetical unreliable when its delay difference is lower than the error boundary $\epsilon$ and results in $g_i = 0$. %+
This relies on the facilitation of a low delay difference value being changed by noise, which results in a different sign of the value and thus a different response, as outlined in Sec. \ref{sec:arbiter}. 

The error boundary $\epsilon$ is a flexible limit, which determines the separation, and is approximated in the attack.
Its value is approximated through the covariance matrix adaptation similar to the way the model values are approximated, as presented in the next Sec. \ref{sec:cma-esdesign}.
Hence $g = (g_1, ..., g_v)$ is the hypothetical reliability vector of $v$ evaluated challenges from the training set by every created model of the attack.

% old:
% As last step the Pearson correlation coefficient of both vectors, reliability vector and hypothetical reliability vector, is computed.
% The correlation coefficient represents the linear dependency of the vectors to each other.
% For the attack it is assumed that a higher linear dependency between these vectors means a more well performing model. %+
% The correlation coefficients are computed of the combinations between the one reliability vector and all hypothetical reliability vectors of all models that exist in the current generation of the algorithm.
As last step Becker's fitness function computes the Pearson correlation coefficient between the reliability values $\gls{h}_i$ and the hypothetical reliability values $g_i$ of all challenges evaluated by the attacked \puf.
The number of evaluated challenges by the attacked \puf is the size of the training set, which is outlined in Chap. \ref{cap:mla}.
The correlation coefficient represents the linear dependency of the vectors to each other.
For the attack it is assumed that a higher linear dependency between these vectors means a more well performing model. %+
The attacked \puf is approximated by adapting its reliability behavior of the responses of every evaluated challenge.
The correlation coefficients are computed of the combinations between the one reliability vector and all hypothetical reliability vectors of all models that exist in the current generation of the algorithm.

These correlation coefficients are used as the fitness values of the corresponding model.
How many models in the current generation of the algorithm exist is explained in the next Sec. \ref{sec:cma-esdesign}.

%========================================

\subsection{CMA-ES}
\label{sec:cma-esdesign}

After we described in the previous section how to assess the fitness values of the created models this sections states how the models are trained by the algorithm.
A pseudocode representation of the \ac{CMA-ES} algorithm is displayed in Alg. \ref{alg:cma-es}.
In this section we analyze the algorithm step by step and point out the purpose of each step. %+
In lines 1 to 3 multiple values are initialized, which are described when used in the algorithm.
Since the \ac{ES} algorithm is based on the natural evolution process survival of the fittest, as described in Sec. \ref{sec:evolutionstrategies}, the algorithm creates new models every generation. % https://en.wikipedia.org/wiki/Survival_of_the_fittest

% Nils: Maybe it is better to remove the initialization from the pseudocode altogether
\SetAlCapHSkip{0.2em}
\begin{algorithm}[H] % t let them float - H argument forces the algorithm to stay in place
\Indm
\SetAlgoLined
\caption{reliability based \acl{CMA-ES} attack}
\label{alg:cma-es}
% \KwData{}
\KwResult{mean values and rates model of the last generation}
\Indp

initialization of the number of new models per generation\\
initialization of the values used for the \ac{CMA}:\\
\ \ \ means, sigma, covariance matrix\\
\While{no termination criteria is fulfilled}{
\For{number of new models per generation}{
create new model from the values:\\
\ \ \ means, covariance matrix, sigma, and random variables\\
rate the fitness of the model by the fitness function\\
}
sort models by their fitness values\\
update the values used in the \ac{CMA}:\\
\ \ \ means, covariance matrix, sigma\\
}
\end{algorithm}

This process of repeating generations is realized by the loop starting in line 4 and ends with fulfilling one of the termination criteria that are described later.
In every generation the following steps are executed.

By the loop starting in line 5 new models are created and their fitness values are determined by the reliability based fitness function, explained in Sec. \ref{sec:reliability}. %+
The number of new created models per generation is defined by the initially set number of new models per generations in line 1.
After all models are created and their fitness rated, they are sorted descending by their fitness values in line 10.

Since the \ac{ES} algorithm requires a selection process of the fittest models in every generation, this is done by the \ac{CMA}.
The \ac{CMA} adapts the values that are used for creating new models in the next generation by the values of models of the current generation in line 11 and 12. %+
The descending sorted models of the current generation are combined with weights to rank their importance for adapting these values.
The impact of a model on the values, used to create the models of the next generation, depends on its fitness value through these weights.
This makes sure that the algorithm approximates fitter models.

The values that are approximated by the \ac{CMA} are the means of the distributions of the values that define a model.
These means in combination with random values, the covariance matrix, and the sigma value are used to create the values for the models of every generation in line 6 and 7. %+
% The covariance matrix represents the correlation of the model values to each other, which is important, as the response of the \apuf depends on every value it is defined by. %+
The covariance matrix represents the correlation of the model values to each other.
The model values determine the response of the \apuf model, as outlined in Sec. \ref{sec:theoretical}, and thus are the values the algorithm learns.
In addition, the error boundary $\epsilon$ that is stated in Sec. \ref{sec:reliability} is approximated by a value of the covariance matrix.

The sigma value is a dynamic step-size value, that adjusts the added random values to create new models.
The value of sigma is reduced when the learning success lowers.
This means it is subsides when the algorithm has correlated a well performing model to increase the likelihood of approximating a better performing model that values differ only slightly to the learned one. %+
Supplementary, sigma sinks when the algorithm can not approximate a model so that the algorithm terminates.

The means, the covariance matrix, and the sigma value are updated every generation in line 11 and 12.
The means are updated by the ranked and weighted models of the current generation. %+
To update the covariance matrix and the sigma value evolution paths are used.
These paths represent correlations between consecutive steps of the modifications of the means. 
Their purpose is to reach a faster approximation and to prevent from early adaption, as explained in Sec. \ref{sec:cma-es}.

The generation loop from line 4 to 13 in the Alg. \ref{alg:cma-es} stops with meeting one of the termination criteria.
The algorithm uses the following three termination criteria:

\begin{itemize}
\item Maximum limit of generations reached
\item Minimum limit of sigma
\item Maximum fitness limit exceeded by at least one trained model
\end{itemize}

The first criteria exits the algorithm after a given number of generations.
The attack ends also if the the step-size sigma becomes to low and the algorithms has approximated a solution or can not approximate a solution in this run. %+
Termination criteria three pertains when a model exceeds a given maximum fitness value.
Hence, at least one of the trained models has to be rated above this given fitness value.

After the algorithm terminated, the result are the approximated means and the rated models of the last generation.
From this point on, the means or the models itself can be used to evaluate challenges of the test set to measure the performance of the trained models. 
If a high proportion of the challenges of the test set is evaluated correct, the reliability based \ac{CMA-ES} attack will have successfully attacked an \apuf. %+
Both the means and the models of the last generation can be used, whereas the means are expected to be better trained then the best trained model of the last generation \cite{2017CMA-ES}.

How high this proportion has to be depends on how many correct evaluated responses are necessary to impersonate the \puf.
However, this depends on the protocol that is based on the \puf and is therefor not part of this thesis.

%========================================

\subsection{Attack combination}
\label{sec:attackcombination}

The reliability based \ac{CMA-ES}, as explained in the previous sections, is applied to \apufs and \mpuf.
To apply the attack to large \xpufs and \mxpuf it has to be performed multiple times in combination with a \ac{CMA-ES} attack that is based on the responses of the \puf \cite{Becker2015ThePUFs}.

One way to display the combined attacks is Alg. \ref{alg:twostageapproach}.
For a better understanding we divide the attack in two stages, whereas line 3 to 5 contains stage one and line 7 to 9 stage two.
Before the attacks start a training set and its reliability values, as clarified in Sec. \ref{sec:reliability}, are created by the \xpuf in line 1.

\SetAlCapHSkip{0.2em}
\begin{algorithm}[H] % t let them float - H argument forces the algorithm to stay in place
\Indm
\SetAlgoLined
\caption{combined attacks on \xpufs}
\label{alg:twostageapproach}
% \KwData{}
\KwResult{mean values and rates model of the last generation}
\Indp

create training set and reliability values by attacked \xpuf\\
\While{not reached number of trained models by stage one}{
reliability based \ac{CMA-ES} attack
}
\While{not approximated a well performing model}{
response based \ac{CMA-ES} attack
}

\end{algorithm}

In stage one the purpose of the reliability based \ac{CMA-ES} attack is to approximate underlying \apufs of the \xpuf one by one.
This can be done, since the reliability of the challenges evaluated by the \xpuf depends "equally on each of the n employed Arbiter PUFs" as Becker et al. mentions \cite{Becker2015ThePUFs}. %+
Hence, the unreliable response of one underlying \apuf results in an unreliable response of the complete \puf.
Because of that, the reliability values measured of the \xpuf can be used to attack the underlying \apufs with the reliability based \ac{CMA-ES} attack, as described in Sec. \ref{sec:cma-esdesign}.

One execution of the reliability based \ac{CMA-ES} attack approximates one of the \apufs.
Multiple executions of the attack lead to different models of underlying \apufs. %+
As the attack is not deterministic, it can approximate models that have a response behavior similar to already approximated models.
These models are eliminated by the hamming distance of their responses and the responses of already approximated models. %+
This approach makes it possible to approximate models for different \apufs used for a \xpuf.
Thus, the reliability based attack is repeated by the loop in line 2 till a given number of trained models is reached.

These models, approximated by the attack in stage one and defined by their mean values, are past on to the stage two.
To be able to combine all approximated models of \apuf to a \xpuf model, the attack of the stage two approximates all the rest of the so far not approximated models. %+
In contrast to the attack in stage one its fitness function scores the models based on their responses of the challenges from the training set. %+
This response based \ac{CMA-ES} attack is repeated by the loop of line 7 in Alg. \ref{alg:twostageapproach} till a model is approximated that evaluates a high proportion of the challenges of the training set correct.

The results of the attack, similar to the normal \ac{CMA-ES} attack, are the means and the approximated models of the last generation.
Both the means or the models can be used to evaluate the challenges of the test set, whereas the means are expected to be better trained \cite{2017CMA-ES}.
If they evaluate a high proportion of the test set challenges correct, the attack will be successful and a well performing model has been approximated. %+

As described in the end of Sec. \ref{sec:cma-esdesign}, the value of this proportion is not part of this thesis.

%========================================
% 1. reliability approach / Fitness:
% reliability based: becker!!!!
% As all known Arbiter PUF implementations suffer from a portion of challenges that do not provide stable responses [10], Why-Attacks-Lose

% Then the reliability hi is computed for challenge blub using the following formula: becker
% Definition: reliability $\gls{h}$ \cite{Becker2015ThePUFs}

% The number of evaluation $\gls{j}$ is the number of times a challenge is evaluated by the same \apuf and its responses $\gls{r}$ captured to calculate it reliability $\gls{h}$.
% Against the definition of stability \ref{equ:stability}, in terms of the reliability based \ac{CMA-ES} attack, the reliability has a different definition.
% The reliability $\gls{h}$ of a challenge $\gls{c}$ is computed of its responses $r_1, r_2, ..., r_j$ by

% \begin{align}
% h &= |\frac{j}{2} - \sum_{i = 1}^{j}r_i|. \label{equ:reliabilityblub}
% \end{align}

% 2. ES and CMA section:
% explain Sigma and what it limits!

% When attacking a puf it does not matter if the rated puf models are sorted descending or ascending as the responses are just boolean values.
% when using the described fitness function
% depending on wether the alg want so find a minimum or a maximum
% minimum search

% 3. attacking \mpuf and \mxpuf differences:
% normal CMA-ES

% two stage CMA-ES
% pseudo code



\chapter{Introduction}

The number of used electronic devices connected over the internet grew throughout the last two decades rapidly.
Through the establishment of smartphones tremendous mass of people are using services based on an internet connection.
Apart form connected end-user devices, control units of industrial facilities or plants that provide any kind of supply can be operated via an internet connection.
These are just some examples for connections where authentication and identification is needed to restrict the access to the qualified entity.
Without restriction every other entity has access and can perhaps enrich itself or cause damage.
Since data, transferred by internet, passes multiple nodes and can be eavesdropped, encryption is needed to protect its containing information.
Authentication, identification, and encryption are just some security features needed to meet the requirements of information security.

To meet the requirements of information security secret keys are used.
Secret keys have to be accessible by the devices to verify themselves and have to be protected.
Approaches like noninvasive memory or one-way hash functions have been used so far.
Though due to the development of physical intrusion methods and the increasing performance of computer systems both techniques facing a number of challenges.
For that reason \pufs, as a new possible basis for secret keys, has been introduced.
\pufs were developed being a alternative to physical implementations of one-way functions used as authentication device such as smart cards. % \cite{Pappu2002PhysicalFunctions}

% old:
% Hence they are mostly stored in data, which makes them vulnerable to being red out and reused by unqualified entities.
% To protect keys one-way hash functions are.
% These functions are easy to evaluate in the one direction but infeasible to calculated in the reverse direction.
% The secret key is applied to the one-way hash function and only its result is stored.
% Hence the secret key can not be calculated from the stored value but conversely.
% However one-way function face 

However \pufs are also not invulnerable to attacks, e.g. physical intrusion or \acl{ML} attacks.
Since \pufs are not completely studied new types of \pufs and suitable attacks are proposed.
This interplay between developing new \pufs that are claimed to be secure and developing of successful attacks leads to a huge variety of different \pufs.
In this way the \apuf and multiple modified versions based on the \apuf have been suggested.
One of these versions is the \mpuf.
As there exist several successful attacks on \apufs, their impact on \mpuf is unknown. % ? probably
Hence this thesis studies the changes due to the modification done to the \apuf and how existing attacks are affected.

The thesis is structured as follows: 
Chap. \ref{cap:background} gives an introduction to \pufs and \ac{ML} attacks.
A more detailed description of important \ac{ML} attacks for this thesis is provided by Chap. \ref{cap:mla}.
\todo{put attack section ref in here}
% old
% Chap. \ref{cap:arbiter} explains the \apuf specifications before Chap. shows already applied attacks occurring in the current literature.
% It is explained why only one of these attacks is relevant to the idea of \ac{MV} which is introduced in Chap. \ref{cap:majorityarbiter}. % and counteracts this one attack.
Chap. \ref{cap:arbiter} explains the \apuf specifications before Chap. shows the modifications made to create the \mpuf.
After an overview to already applied attacks occurring in the current literature it is explained why only one of these attacks is relevant to the idea of \ac{MV}.
To prove this in an empirical manner a simulation of the \apuf and the attack is build and described in Chap. \ref{cap:simulationdesign}.
The results of the simulations are laid out in Chap. \ref{cap:stabilitysimulation} and Chap. \ref{cap:attacksimulations}.
Finally a conclusion for the thesis is given in Chap. \ref{cap:conclusion}.

% iot
% no security due to expenses (iot)
% bot nets
% ==============================
% - smart phones, cars, smart -everything
% - authentication
% - identification
% - encryption: for data, connections, signatures
% - attacks to steal secret keys
% - noninvasive memory
% - hash functions 
% - PUFs provide a alternative key storage developed to resist these attacks
% - (upon pufs Protocols are based )
% - gegenspiel zw attack and puf dev -> arbiter -> improvement by mv
% - Machine learning attacks on majority vote arbiter physical unclonable functions.

\chapter{Machine Learning Attacks}
\label{cap:mla}

Different \ac{ML} attacks are used to attack \pufs.
This section gives an overview of the techniques used in these attacks including the attack that is examined in this thesis. %+
First, the required inputs for the \ac{ML} attacks and the problem they are solving are explained.
After that, each \ac{ML} technique is outlined by its own.

The \ac{ML} attack creates a \puf model of the attacked \puf, which acts as binary classifier.
A binary classifier divides the samples, in the case of \pufs these are challenges, of a given set into two different classes $0$ or $1$ similar to the responses of a \puf. %+
% old:
% To build this model a training set $D = \{(z, e)_1, ..., (z,e)_i\}$ with samples $d = (z, e)_i$ that are already classified by the attacked \puf are required.
% A sample $d$ is defined by its values $z$ and is classified into the class $e$. %+
To build this model a training set $D = \{(z_1, e_1), ..., (z_i,e_i)\}$ is necessary, whereas the integer $i$ is the number of training set samples $d_i = (z_i, e_i)$ that are already classified by the attacked \puf.
Let $l$ be the number of input values of a sample $d_i$.
This sample is then defined by its values $z_i = (z_{i,1}, ...,z_{i,l})$ and is classified into the class $e_i$. 
%+

With that training set the model can be trained by the \ac{ML} algorithm.
To evaluate the built model, a test set is created the same way and used to compare the output of the \puf model and the attacked \puf.

The most \pufs, known by the year 2014, which can be built by an electrical circuit, are related to internal values of the circuit \cite{Ruhrmair2014PUFOverview}.
These values determine the response for a given challenge. %+
The goal of the \ac{ML} attacks is to find a representation of these values, so that the created model classifies samples in the same way as the attacked \puf. % values b
How man samples of the test set the model classifies the same way as the \puf defines how precise the model has been trained. %+
Accordingly, the values learned by the \ac{ML} attack do not have to be exact the same values that defined the attacked \puf to classify samples of a set the same way as the \puf.

Whether a \ac{ML} attack can find a binary classifier for a given set depends on the attack and if the set of samples is linear separable. 
Instead of a set of samples this also applies to binary functions $f: \mathbb{R}^n \to \{0,1\}$. %+ % https://de.wikipedia.org/wiki/Lineare_Separierbarkeit
One example for a not linear separable function is the \ac{XOR} function. % https://de.wikipedia.org/wiki/Lineare_Separierbarkeit
Some \ac{ML} attacks, e.g.\ single-layer perceptron, can only learn linear separable sets and can therefore not learn \ac{XOR} functions \cite{Minsky1969Perceptrons:Geometry}. %+
Two subsets $A \subseteq \mathbb{R}^n$, $B \subseteq \mathbb{R}^n$ are called linear separable if all their vectors $\vec{a} = (a_1, ..., a_n) \in A$, $\vec{b} = (b_1, ..., b_n) \in B$ can be separated by

\begin{align}
\sum_{i=1}^{n} w_i a_i \le w_{n + 1} < \sum_{j=1}^{n} w_j b_j, \label{equ:linearseparable}
\end{align}

whereas $w_1, ..., w_{n + 1}$ are $n + 1$ real numbers.
The separating hyperplane is defined by the vector $\vec{x} = (x_1, ..., x_n) \subseteq \mathbb{R}^n$ that satisfies $\sum_{i = 1}^{n} w_i x_i = w_{n + 1}$.% https://en.wikipedia.org/wiki/Linear_separability
In the case of building a classifier by the \ac{ML} attacks the two subset $A$ and $B$ are subsets of the given set used to train the model. %+
The \ac{ML} attacks named in this chapter are common approaches to attack \pufs \cite{Ruhrmair2014PUFOverview}.
As there are different modifications and configurations used, only the \ac{ML} algorithms itself are stated.

% over trained / over fitted model (end second paragraph)
% ? randomly chosen values are most of the time normal distributed
%========================================
% values z of the samples are a real-valued vectors % https://en.wikipedia.org/wiki/Perceptron
% classifications e are boolean
% done
% -All by the year 2014 know \cite{Ruhrmair2014PUFOverview} Page 4
% - All described \ac{ML} are binary classifier 
% -what is a binary classifier % https://en.wikipedia.org/wiki/Binary_classification
% -to train the \ac{ML} attack
% -training set: $D = \{(z, e), ..., (z,e)\}$ contains samples d with their values z and their correct classification e
% -to test the \ac{ML} attack test set $T...$
% -Definition of linear separability! % https://de.wikipedia.org/wiki/Lineare_Separierbarkeit
% 
% tmp-old:
% Two subsets $A \subseteq \mathbb{R}^n$, $B \subseteq \mathbb{R}^n$ are called linear separable if all their vectors $\vec{a} = (a_1, ..., a_n) \in A$, $\vec{b} = (b_1, ..., b_n) \in B$ can be separated by $n + 1$ real numbers $w_1, ..., w_{n + 1}$ and therefore
% 
% \begin{align}
% \sum_{i=1}^{n} w_i a_i \le w_{n + 1} < \sum_{j=1}^{n} w_j b_j \label{equ:linearseparable}
% \end{align}
% 
% applies.

\section{Evolution Strategies}
\label{sec:evolutionstrategies}

The \acf{ES} is an optimization technique that is inspired by the evolution process of the nature.
In this process every species can adapt to the environmental conditions by changing its genes and only the fittest survive.
From these fittest individuals the next generation will be built and inherits the attributes of their fittest parents genes. %+

This repeating process of classification by rating the fitness of every individual, inheriting the genes of the previous generation and modifying them leads to an approximation of the optimal solution for the specific environment.
Every repetition of this evolutionary cycle counts as one generation and every individual represents a model with its own fitness value	\cite{Becker2015ThePUFs}. % wiki

The model that is rated to be the fittest model after the last generation is the result of the algorithm and the best learned model by this time.
It is also possible to approximate the at least optimal solution by letting the at least fittest survive. %+
The procedure of the \ac{ES} attack, in the case of finding the optimal solution, is displayed in the Alg.\ \ref{alg:es}.

% \IncMargin{1em}
\SetAlCapHSkip{0.2em}
\begin{algorithm}[ht] % t let them float - H argument forces the algorithm to stay in place
\Indm
\SetAlgoLined
\caption{\acl{ES}}
\label{alg:es}
% \KwData{}
\KwResult{sets of models of the last generation}
\Indp

initialization of the models of the first generation\\
\While{termination criteria is not fulfilled}{
modify all models\\
rate fitness of all models\\
sort all models by their fitness values\\
select number of the fittest models\\
create new generation from selected models\\
}
\end{algorithm}
% \DecMargin{1em}

To initialize the models of the first generation in line $1$, often random chosen normally distributed values are used.
In the step of the lines $6$ and $7$ of the attack, it is determined how the number of models changes in the evolution process. %+
There are options to define how many of the fittest models are selected and how many new models are created per generation.

Additionally, in the step of creating the next generation it can be defined if models of the old generation should be kept and if new randomly built models should be added. %+
Adding new random models makes it possible for the attack to approximate a more optimal solution even if all models are already approximated to another solution \cite{Rechenberg1994EvolutionsstrategiePrinzipien}.
Keeping old models ensures not to lose an individual of the previous generation, which outperforms the next generations models.

How the models are modified in the beginning of every generation in line $3$ of the Alg.\ \ref{alg:es} is not specified.
One way of modification is to add a random chosen value to all values of the individual. %+
The fitness function in line $4$ rates the fitness value of every individual and represents the above mentioned environmental conditions that the models get adapted to. %+

The sort function of line $5$ sorts the results by their fitness values.
The \ac{ES} attack approximates an optimal solution if the models are sorted descending and vice versa. %+
An attack ends when a termination criteria is fulfilled, which could be e.g.\ a maximum number of generations.

The \ac{ES} attack is a randomized algorithm and therefore does not always determines.
Nevertheless, it can approximate not linear separable problems and does not need a differentiable model \cite{Ruhrmair2010ModelingFunctions}.
\pagebreak

%========================================
% The computation time required by ES is determined by the following factors: \cite{Ruhrmair2013PUFData}
% ES is a randomized method that neither requires an (approximately) linearly separable problem (like Support VectorMachines), nor a differentiable model (such as LR with gradient descent); a merely param- eterizable model suffic. \cite{Ruhrmair2010ModelingFunctions}
% notation: with / without parents: page 25
% \cite{Becker2015ThePUFs}
% https://de.wikipedia.org/wiki/Evolution%C3%A4rer_Algorithmus
% http://www.scholarpedia.org/article/Evolution_strategies

\section{CMA-ES}
\label{sec:cma-es}

The \acf{CMA-ES} is based on the principles of the \ac{ES}, which is shown in Sec.\ \ref{sec:evolutionstrategies}.
Additional, the \ac{CMA-ES} algorithm uses a covariance matrix to learn a second order model of the underlying function.

For this reason every value that defines the model and is learned by the algorithm has its own distribution, which will be optimized.
Hence, the covariance matrix is used to represent pairwise dependencies between variables used in these distributions. %+
% Throughout the evolution process the mean values and the covariance matrix used in combination with normal distributed random distributed values to generate the next generation of models and are adaped iterative in every generation.
Throughout the evolution process the mean values of the distributions, the covariance matrix and a step size value are adapted per generation. %+

The mean values, the covariance matrix, and the step size in combination with normal distributed randomly chosen values are used to create the model for the next generation in every evolution cycle \cite{Hansen2011TheTutorial}.
To update the values used in the \ac{CMA-ES} algorithm two principles are applied.

The first principle is the maximum-likelihood method.
It is used to maximize the likelihood of previous models with the highest fitness values when updating the mean values.
When updating the covariance matrix, the likelihood of previous search steps that lead to fitter models is increased.
% principal components analysis

The second principle captures two paths of the time progress of the distribution mean values.
These two paths are called evolution paths.
They are used to gaining information on correlations between consecutive steps of the mean values modification.
The evolution paths becomes longer when consecutive step are taken in the same direction. %+
The \ac{CMA} process uses the first path to promote longer paths, which represent preferred directions of the distribution mean values modification, by the possibility of a faster distribution variance increase. %+

To control the step size value the second evolution path is used.
The step size is adjusted by \ac{CSA} that reduces the importance of a step with the increasing number of taken steps \cite{Chotard2012CumulativeFunctions}.
This allows the algorithm to approximate the optimal solution fast and prevents the \ac{CMA-ES} algorithm from to early adaptations. 

The \ac{CMA} is built to solve non-separable and ill-conditioned problems \cite{Hansen2011TheTutorial}.
Problems are ill-conditioned when making little changes to their input lead to significant changes in their output \cite{Belsley1980RegressionCollinearity}.

% principal components analysis
% An n-dimensional separable problem can be solved by solving n 1-dimensional problems separately, which is a far easier task.
%========================================
% Not used:
% \cite{Hansen2006TheReview}
% https://homepages.fhv.at/hgb/downloads/cma-es.mat
% https://www.lri.fr/~hansen/copenhagen-cma-es.pdf

\section{Support Vector Machine}
\label{sec:svm}

A \acf{SVM} is the model created by the \ac{SVM} training algorithm.
It classifies samples into two categories with dividing their multidimensional space by a hyperplane. % hyperplane has dimension z - 1

The number of dimensions of the space is defined by the number of values $z$ of the samples.
The model is trained the way that it divides the two categories by the largest possible gap. %+
To achieve this, the training algorithm maximizes the distance from the hyperplane to the nearest samples of each category.
If there exists such a hyperplane, known as maximum-margin hyperplane, it will have the larges margin to the closest samples of both categories. %+

Not all samples have to be considered when training the model.
Samples that are located behind other samples of the same classification and are more far away of the hyperplane do not affect the location of the hyperplane, since they never become the closest samples. %+
New samples are then predicted to belong to the same category as the samples that are on the side of the hyperplane where the new sample is located.

A complete separation into two categories by a linear hyperplane is only possible if the training data is linear separable.
To make it possible to classify non-linear problems \acp{SVM} use the kernel method. %+
In doing so, all training data and the $z$ dimensional space is transformed to a space with a higher number of dimensions.
All data transformed into a space with enough higher number of dimensions is linear separable.
That can be even an infinite number of dimensions. %+
The hyperplane is then determined in this higher dimensional space and transformed back to the original dimensional space where it represents a non-linear hyperplane. %+
As these transformations are high in computational cost, the kernel method provides the possibility to define the hyperplane in a higher dimensional space and use it in the lower dimensional space without the transformations \cite{Cristianini2000AnMethods}.

Another method used by \ac{SVM} to be able to solve non-linear problems are slack variables.
Slack variable allow the algorithm by undergoing punishment to classify samples in the training process wrong. %+
This leads to a more flexible training process, reduces the number of needed samples and can build models of non-separable data.

% Soft Margin Classifier
% In practice, real data is messy and cannot be separated perfectly with a hyperplane.
% The constraint of maximizing the margin of the line that separates the classes must be relaxed. This is often called the soft margin classifier. This change allows some points in the training data to violate the separating line....
%========================================
% https://www.youtube.com/watch?v=1NxnPkZM9bc
% Source:
% kernel method https://en.wikipedia.org/wiki/Kernel_method
% Die Idee hinter dem Kernel-Trick ist, den Vektorraum und damit auch die darin befindlichen Trainingsvektoren in einen höherdimensionalen Raum zu überführen. In einem Raum mit genügend hoher Dimensionsanzahl – im Zweifelsfall unendlich – wird auch die verschachteltste Vektormenge linear trennbar. In diesem höherdimensionalen Raum wird nun die trennende Hyperebene bestimmt. Bei der Rücktransformation in den niedrigerdimensionalen Raum wird die lineare Hyperebene zu einer nichtlinearen, unter Umständen sogar nicht zusammenhängenden Hyperfläche, die die Trainingsvektoren sauber in zwei Klassen trennt....
% https://en.wikipedia.org/wiki/Support_vector_machine
% https://de.wikipedia.org/wiki/Support_Vector_Machine


\section{Logistic Regression}
\label{sec:lr}

The \acf{LR} is a regression analysis technique, which is used in statistic modeling to estimate relationships between variables. % https://en.wikipedia.org/wiki/Regression_analysis https://en.wikipedia.org/wiki/Logistic_regression
It is an approach to solve binary classification problems and is based on the logistic function $\frac{1}{1 + e^{-z_{i,l}}}$.

The logistic function is a ``S'' shape and its values ranges between $0$ and $1$.
The input value $z_{i,l}$ is transformed onto this range, since it is applied in the denominator of the function. %+
% In \ac{LR} a modified version of the logistic function is used to transform all values that are going to be classified by 

In the \ac{LR}, additionally to the logistic function, all values $z_i = (z_{i,1}, ...,z_{i,l})$ of a sample that is going to be classified are combined with weights $\beta =(\beta_1, ..., \beta_l)$ and a bias $\beta_0$ is added. %+
Hence, this modified version of the logistic function, known as \ac{LR} model, is used to transform all values $z_i$ by 

\begin{align*}
\Pr[Y = 1 | Z = z_i] &= \frac{1}{1 + e^{-(\beta_0 + z^T_i \beta)}},
\end{align*}

onto the range between $0$ and $1$.
% The value of the function is the likelihood $P$ that a given sample with its values $z_i$ is classified with $Y = 1$.
The value of the function is the likelihood $\Pr[Y = 1 | Z = z_i]$ that a given sample classifies $1$ under the condition its values are $z_i$.
To calculate the binary classification, the likelihood is then divided into the classes $0$ for $\Pr[Y = 1 | Z = z_i] < 0.5$ and $1$ for $\Pr[Y = 1 | Z = z_i] \ge 0.5$. %+

The \ac{LR} model is defined by the weights $\beta + \beta_0$. % whereas the bias $\beta_0$ is the value which determines the likelihood $\Pr[Y = 1 | Z = z_i]$ when all values $z_i$ are zero. % https://en.wikipedia.org/wiki/Logistic_regression
These weights must be estimated from the training data that includes already classified samples.
A usual approach is the maximum likelihood estimation whereas the goal is that the likelihood, predicted by the \ac{LR} model, is very close to $1$ if the sample belongs to this class and the way round. %+

The maximum likelihood estimation does that by minimizing the error in the likelihood predicted by the model for the training data. % Probably cut out http://machinelearningmastery.com/logistic-regression-for-machine-learning/
For more detailed information I refer the reader to \cite{Aldrich1997RA1922}.

As a last point is has to be mentioned that the classification problems solved by \ac{LR} do not need to be linearly separable but differentiable \cite{Ruhrmair2010ModelingFunctions}.
% https://en.wikipedia.org/wiki/Logistic_regression

%========================================
% values z are called feature vektor 
% Sources:
% http://machinelearningmastery.com/logistic-regression-for-machine-learning/
% addition source from ModelingFunctions: C.M. Bishop et al. Pattern recognition and machine learning. Springer New York:, 2006

\section{Perceptron}
\label{sec:perceptron}

The perceptron is a supervised learning algorithm of binary classifiers and the simplest form of an \ac{ANN} \cite{Rosenblatt1957TheAutomaton}. 

In case of a single-layer perceptron the values $z_i$ of a sample $d_i$, which is going to be classified, are combined with the weights $\beta =(\beta_1, ..., \beta_l)$ and a bias $\beta_0$ is added, similar to \ac{LR} shown in Sec.\ \ref{sec:lr}.
However, the output value $f(z_i)$ of the perceptron is calculated by

\begin{align*}
f(z_i) &= \begin{cases}
1 & \text{if}\ \beta \cdot z_i + \beta_0 > 0\\
0 & \text{otherwise}
\end{cases}
\end{align*}

and classifies the sample $d_i$ in the classes $0$ or $1$.
To obtain the weights and the bias, which define the perceptron model, the model is trained by the Alg.\ \ref{alg:perceptron} and a training set with the samples $d_i$ and their desired output $e_i$.

\SetAlCapHSkip{0.2em}
\begin{algorithm}[H]
\Indm
\SetAlgoLined
\caption{perceptron}
\label{alg:perceptron}
\KwData{training set}
\KwResult{weights and bias} %weights $\beta$ and bias $\beta_0$
\Indp

initialize the weights\\ % $\beta$\\
initialize the bias\\ % $\beta_0$\\
\For{each sample of the training set}{ 
calculate the output of the perceptron\\
update the weights and the bias\\
}

\end{algorithm}

The output value of the perceptron $f(z_i)$ is computed for a sample $d_i$ with $\beta(p)$ and $\beta_0(p)$ at the time $p$ in line $4$.
Afterwards, in line $5$ the weights $\beta$ and the bias $\beta_0$ are updated by

\begin{align*}
\beta_i(p+1) &= \beta_i(p) + \alpha \cdot (e_i - f(z_i)) \cdot z_i,\\
\beta_0(p+1) &= \beta_0(p) + \alpha \cdot (e_i - f(z_i)),
\end{align*}

whereas $\beta_i(p)$ are the weights $\beta_i$ and $\beta_0(p)$ is the bias $\beta_0$ at the time $p$.
A learning rate $\alpha$ is added that determines the value that is used to change the weights.

With these incremental updates the perceptron algorithm is an online learning algorithm.
The algorithm is guaranteed to converge, and there is a limit for the maximum number of weight adjustments while trained under the condition that the training set is linear separable \cite{Rosenblatt1957TheAutomaton}. %+

For a non-linear separable problem the perceptron algorithm does not terminate as it never reaches the optimal classification of all samples. % https://en.wikipedia.org/wiki/Perceptron
A modification of the single-layer perceptron is the multilayer perceptron that is able to distinguish non-linear separable problems.
For more detailed information about the multilayer perceptron I refer the reader to \cite{Rosenblatt1961Principlesmechanisms}.


%========================================
% Sources:
% can learn non-linear function if not single layered % https://en.wikipedia.org/wiki/Perceptrons_(book)
% multilayer perceptrons % https://en.wikipedia.org/wiki/Multilayer_perceptron
% -> backpropagation % https://en.wikipedia.org/wiki/Backpropagation
% for non-linear: dalta rule % https://en.wikipedia.org/wiki/Delta_rule

% \section{Artificial Neural Network}
% \label{sec:neuralnetwork}

% Artificial Neural Networks (ANN): ANNs are \cite{Hospodar2012MachineUsability}

% a special form of ANN is perceptron learning

%========================================

% \section{Low-Degree Algorithm}
% \label{sec:lowdegreealg}

%========================================

% \section{PAC Learning}
% \label{PAC learning}

% PAC Learning of XOR PUFs \cite{Ganji2015WhyPUFs}

%========================================
